%version of 11 Nov 2019
\documentclass[12pt]{article}
%\usepackage{psfig, amssymb}
\usepackage{amsfonts,amsmath,latexsym,amssymb,mathrsfs}
%%%%for breaking citation so that they don't overrun
\usepackage{breakcites}
%%%%
%\usepackage{refcheck}%for checking whether equation numbers and references are used.
\usepackage[pdftex]{graphicx}
\usepackage{wrapfig}
\usepackage{caption}
\usepackage{color}

\usepackage{geometry}
\usepackage{enumerate}
%\graphicspath{ {./images/} }
\usepackage[T1]{fontenc}
\usepackage[english]{babel}
\usepackage{dsfont,bm,bbm,mathrsfs}
\usepackage{stmaryrd}
\usepackage{url,color}
\usepackage{textcase}
%\usepackage{showlabels}
\usepackage{multirow}
\usepackage{tikz}
\usetikzlibrary{intersections}

\allowdisplaybreaks

%Shortcuts for Theorems


\newtheorem{thm}{Theorem}[section]
\newtheorem{prop}[thm]{Proposition}
\newtheorem{defi}[thm]{Definition}
\newtheorem{lma}[thm]{Lemma}
\newtheorem{cor}[thm]{Corollary}
\newtheorem{exam}[thm]{Example}
\newtheorem{countexam}[thm]{Counterexample}
\newtheorem{rem}[thm]{Remark}
\newtheorem{con}[thm]{Conjecture}



\makeatletter
%%% Define shorter display equations commands
\def\be#1{\begin{equation*}#1\end{equation*}}
\def\ben#1{\begin{equation}#1\end{equation}}
\def\bes#1{\begin{equation*}\begin{split}#1\end{split}\end{equation*}}
\def\besn#1{\begin{equation}\begin{split}#1\end{split}\end{equation}}
\def\bea#1{\begin{align*}#1\end{align*}}
\def\bean#1{\begin{align}#1\end{align}}
\def\bg#1{\begin{gather*}#1\end{gather*}}
\def\bgn#1{\begin{gather}#1\end{gather}}
\setlength{\multlinegap}{1em}
\def\bml#1{\begin{multline*}#1\end{multline*}}
\def\bmln#1{\begin{multline}#1\end{multline}}

\def\note#1{\par\smallskip%
\noindent%
\llap{$\boldsymbol\Longrightarrow$}%
\fbox{\vtop{\hsize=0.98\hsize\parindent=0cm\small\rm #1}}%
\rlap{$\boldsymbol\Longleftarrow$}%
\par\smallskip}


%%% Define bracket commands
\def\given{\mskip 0.5mu plus 0.25mu\vert\mskip 0.5mu plus 0.15mu}
\newcounter{@bracketlevel}
\def\@bracketfactory#1#2#3#4#5#6{
\expandafter\def\csname#1\endcsname##1{%
\addtocounter{@bracketlevel}{1}%
\global\expandafter\let\csname @middummy\alph{@bracketlevel}\endcsname\given%
\global\def\given{\mskip#5\csname#4\endcsname\vert\mskip#6}\csname#4l\endcsname#2##1\csname#4r\endcsname#3%
\global\expandafter\let\expandafter\given\csname @middummy\alph{@bracketlevel}\endcsname
\addtocounter{@bracketlevel}{-1}}%
}
\def\bracketfactory#1#2#3{%
\@bracketfactory{#1}{#2}{#3}{relax}{0.5mu plus 0.25mu}{0.5mu plus 0.15mu}
\@bracketfactory{b#1}{#2}{#3}{big}{1mu plus 0.25mu minus 0.25mu}{0.6mu plus 0.15mu minus 0.15mu}
\@bracketfactory{bb#1}{#2}{#3}{Big}{2.4mu plus 0.8mu minus 0.8mu}{1.8mu plus 0.6mu minus 0.6mu}
\@bracketfactory{bbb#1}{#2}{#3}{bigg}{3.2mu plus 1mu minus 1mu}{2.4mu plus 0.75mu minus 0.75mu}
\@bracketfactory{bbbb#1}{#2}{#3}{Bigg}{4mu plus 1mu minus 1mu}{3mu plus 0.75mu minus 0.75mu}
}
\bracketfactory{clc}{\lbrace}{\rbrace}
\bracketfactory{clr}{(}{)}
\bracketfactory{cls}{[}{]}
\bracketfactory{abs}{\lvert}{\rvert}
\bracketfactory{norm}{\Vert}{\Vert}
\bracketfactory{floor}{\lfloor}{\rfloor}
\bracketfactory{ceil}{\lceil}{\rceil}
\bracketfactory{angle}{\langle}{\rangle}

%%%Define scr commands
\def\scrF{\mathscr{F}}
\def\scrG{\mathscr{G}}

%%% Define corresponding vertical bar commands

\def\calF{\mathcal{F}}
\def\calG{\mathcal{G}}
\def\ahalf{{\textstyle\frac12}}
\def\eq#1{\eqref{#1}}
\def\I{\mathrm{I}}
\def\Bi{\mathrm{Bi}}
\def\IP{\prob}

\def\ER{Erd\H{o}s-R\'enyi}
\def\hKin{\hat{K}^{\mathrm{in}}}
\def\hKout{\hat{K}^{\mathrm{out}}}
\def\Kin{K^{\mathrm{in}}}
\def\Kout{K^{\mathrm{out}}}

\makeatother


%Shortcuts for sets of numbers
\newcommand{\law}{\mathscr{L}}
\newcommand{\HH}{\mathscr{H}}
\newcommand{\D}{\mathbb{D}}
\newcommand{\Pro}{\mathbb{P}} 
\newcommand{\prob}{\Pro}
\DeclareMathOperator{\E}{\mathbb{E}}
\newcommand{\IE}{\E}
\newcommand{\mean}{\E}
\newcommand{\R}{\mathbb{R}}
\newcommand{\N}{\mathbb{N}}
\newcommand{\non}{\nonumber}
\newcommand{\Z}{\mathbb{Z}}
\newcommand{\C}{\mathbb{C}}
\newcommand{\LL}{\textbf{L}}
\DeclareMathOperator{\Var}{\mathrm{Var}}
\DeclareMathOperator{\var}{\mathrm{Var}}
\DeclareMathOperator{\cov}{\mathrm{Cov}}
\DeclareMathOperator{\bigo}{\mathrm{O}}
\newcommand{\K}{\textbf{Ker}}
\newcommand{\Id}{\textbf{Id}}

%%%%%%%%%%%%%
%%%%%Xia's commands
%%%%%%%%%%%%%
\usepackage{subfig}

\newcommand{\bn}{{\bf N}}
\newcommand{\bone}{{\bf 1}}
\newcommand{\bc}{{\mathds{C}}}
\newcommand{\bp}{{\bf P}}
\newcommand{\bh}{{\textbf H}}
\newcommand{\bi}{{\bf i}}
\newcommand{\bk}{{\bf k}}
\newcommand{\br}{{\bf R}}
\newcommand{\bq}{{\bf Q}}
\newcommand{\bs}{{\bf S}}
\newcommand{\bz}{{\bf Z}}
\newcommand{\bG}{\mbox{\boldmath$\Gamma$}}
\newcommand{\ba}{{\bm\alpha}}
\newcommand{\bl}{{\bm\lambda}}
\newcommand{\bmu}{{\bm\mu}}
\newcommand{\bx}{{\bm x}}
\newcommand{\bX}{{\bm X}}
\newcommand{\bY}{{\bm Y}}
\newcommand{\bZ}{{\bm Z}}
\newcommand{\cF}{{\cal F}}
\newcommand{\ca}{{\cal A}}
\newcommand{\cb}{{\cal B}}
\newcommand{\scrB}{{\mathscr B}}
\newcommand{\cc}{{\cal C}}
\newcommand{\cd}{{\cal D}}
\newcommand{\cf}{{\cal F}}
\newcommand{\cg}{{\cal G}}
\newcommand{\ch}{{\cal H}}
\newcommand{\ci}{{\cal I}}
\newcommand{\cl}{{\cal L}}
\newcommand{\cm}{{\cal M}}
\newcommand{\cn}{{\cal N}}
\newcommand{\cs}{{\cal S}}
\newcommand{\cu}{{\cal U}}
\newcommand{\cy}{{\cal Y}}
\newcommand{\dtv}{{d_{\rm TV}}}
\newcommand{\dk}{{d_{\rm K}}}
\newcommand{\dw}{{d_{\rm W}}}
\def\tg{{\tilde g}}
\def\tf{{\tilde f}}
\newcommand{\Pn}{{\rm Pn}}
\def\t#1{^{(#1)}}

\DeclareMathOperator*{\esssup}{ess\,sup}
\DeclareMathOperator*{\essinf}{ess\,inf}
\newcommand{\Po}{{\rm Pn}}

\def\Ref#1{(\ref{#1})}
\def\a{\alpha}
\def\b{\beta}
\def\s{\sigma}
\def\f{\phi}
\def\w{\omega}
\def\l{\lambda}
\def\La{\Lambda}
\newcommand{\oF}{{\overline{F}}}
\def\p{\pi}
\def\tod{\buildrel{{\rm d}{\to}}}
\def\eqd{\stackrel{\rm d}{=}}
\def\siim{\sum_{i=1}^m}
\def\sjn{\sum_{j=1}^n}
\def\dw{d_{\mathrm{W}}}
\def\dtv{d_{\mathrm{TV}}}
\def\tpj{\overline{\tau^+_j}}
\def\tmj{\overline{\tau^-_j}}
\def\tpi{\overline{\tau^+_i}}
\def\tpi1{\overline{\tau^+_{i-1}}}
\def\tmi{\overline{\tau^-_i}}
\def\stg{\stackrel{\mbox{\scriptsize st}}{\ge }}
\def\stl{\stackrel{\mbox{\scriptsize st}}{\le }}

\newcommand*\rot[1]{\rotatebox{90}{#1}}

\makeatletter

% \newcommand{\qed}{\box}
\newcommand{\qed}{\nopagebreak\hspace*{\fill}
{\vrule width6pt height6ptdepth0pt}\par}


\newcommand{\blue}[1]{\textcolor{blue}{#1}}
\newcommand{\red}[1]{\textcolor{red}{#1}}
\newcommand{\yellow}[1]{\textcolor{yellow}{#1}}
\newcommand{\green}[1]{\textcolor{green}{#1}}
\newcommand{\cyan}[1]{\textcolor{cyan}{#1}}
\newcommand{\magenta}[1]{\textcolor{magenta}{#1}}

\def\ignore#1{}

\def\Comment#1{
    \marginpar{$\bullet$\quad{\tiny #1}}}

%%%%%definition of counters

\makeatletter
\newcommand*\labelcounter[2]{\begingroup
  \protected@edef\@currentlabel{\csname p@#1\endcsname\csname the#1\endcsname}%
  \label{#2}\endgroup}
\newcommand*\refsetcounter[2]{\setcounter{#1}{#2}%
  \protected@edef\@currentlabel{\csname p@#1\endcsname\csname the#1\endcsname}%
  }
\makeatother

\newcounter{rtaskno}
\DeclareRobustCommand{\rtask}[1]{%
   \refstepcounter{rtaskno}%
   \thertaskno\label{#1}}

\ignore{\newcounter{con}
\DeclareRobustCommand{\const}[1]{%
   \refstepcounter{con}%
   \thertaskno\label{#1}}}

\newcounter{con}%for constants
\newcommand{\const}{{\stepcounter{con}\rm\thecon}}
%to use the counter number, type \const and then immediately use \qcon{} so that the counter will be added by 1.
%Use \labelcounter{con}{eq:123} to extract the numeric number of the counter at this point

%%%%%end of definition of counters

%%%%%%%%%%%%%%
%end of Xia's commands
%%%%%%%%%%%%%%

\numberwithin{equation}{section}

\title{\sc\bf\large\MakeUppercase{A revisit of large deviation of Poisson binomial}}

\author{ Qingwei Liu\footnote{}
\ and \
 Aihua Xia\footnote{School of Mathematics and Statistics,
The University of Melbourne,
VIC 3010, Australia, E-mail: aihuaxia@unimelb.edu.au. Work supported in part by Australian Research Council Grants No DP220102666.}
}

\def\parsedate #1:20#2#3#4#5#6#7#8\empty{20#2#3-#4#5-#6#7}
\def\moddate{\expandafter\parsedate\pdffilemoddate{\jobname.tex}\empty}
\date{\moddate}



\begin{document}
\maketitle

\begin{abstract} TBA\end{abstract}


\vskip12pt \noindent\textit {Key words and phrases\/}: Stein-Chen method, Poisson approximation,
moderate deviation.

\vskip12pt \noindent\textit{AMS 2020 Subject Classification\/}:



\section{The properties of the solution}%\label{secIntroduction}

Let $Y$ be a rv having the distribution determined by $\alpha$ and $\beta$, write $\beta_j=\beta j+j(j-1)$, $F(j)=\IP(Y\le j)$, $\oF(j)=\IP(Y\ge j)$, for $j\in \Z_+$. Define 
\begin{equation}\scrB g(i)=\alpha g(i+1)-\beta_ig(i),\label{steinid}\end{equation}
and set the Stein equation as
\begin{equation}
\scrB g(i)=\bone_{[k,\infty)}(i)-\E\bone_{[k,\infty)}(Y)=\bone_{[k,\infty)}(i)-\overline{F}(k)\label{steineq},
\end{equation}
write the solution to \Ref{steineq} as $g_k$. 
Noting that $g(0)$ plays no role in \Ref{steineq}, we set $g(0)=g(1)$. 

\begin{lma}\label{lma1}

{\begin{description}
\item{(i)} $\Delta g_k(i)<0$ for $i\le k-1$ and $\Delta g_k(i)>0$ for $i\ge k$.
\item{(ii)} $\Delta^2 g_k(i)<0$ for $1\le i\le k-2$ and $i\ge k$, and $\Delta^2 g_k(k-1)>0$.
\item{(iii)} For $k\ge 2$, $\Delta^2 g_k(k-1)< \frac{1-\pi_0}{\alpha}\wedge \frac1{\beta_{k-1}}\wedge \frac1{\beta+1}.$
\item{(iv)} $\Delta^3 g_k(i)<0$ for $1\le i\le k-3$ and $i= k-1$, and $\Delta^3 g_k(i)>0$ for $i= k-2$ and $i\ge k$.
\end{description}}
\end{lma}

\noindent{\bf Proof of Lemma~\ref{lma1}} For convenience, we drop the subindex $k$ from $g_k$, and use the representation of \cite{BX01} to argue as ... to get
 \ben{\label{sol1}\Delta g(i)=\begin{cases}
\frac{\oF(k)}{\pi_i}\left(\frac{F(i-1)}{\beta_i}-\frac{F(i)}{\alpha}\right),& \mbox{ for }i\le k-1,\\
\frac{F(k-1)}{\pi_i}\left(\frac{\oF(i)}{\beta_i}-\frac{\oF(i+1)}{\alpha}\right),& \mbox{ for }i\ge k.
\end{cases}}

(i) For $i\le k-1$, since $\alpha F(i-1)=\sum_{j=1}^i\beta_j\pi_j\le \beta_i\sum_{j=1}^i\pi_j<\beta_iF(i)$, we have $\Delta g(i)<0$. Similarly, for $i\ge k$, we have $\alpha\oF(i)=\sum_{j=i+1}^\infty\beta_j\pi_j>\beta_i\oF(i+1)$, we obtain $\Delta g(i)>0$. 

(ii) For $1\le i\le k-2$,
 \begin{align*}\Delta^2 g(i)&=-\oF(k)\left(\frac{F(i+1)}{\alpha\pi_{i+1}}-2\frac{F(i)}{\alpha\pi_i}+\frac{F(i-1)}{\alpha\pi_{i-1}}\right)\nonumber\\
 &=-\frac{\oF(k)}{\alpha^3\pi_{i-1}}\left(\beta_{i+1}\beta_iF(i+1)-2\alpha\beta_iF(i)+\alpha^2F(i-1)\right)\nonumber\\
 &=-\frac{\oF(k)}{\alpha^3\pi_{i-1}}\sum_{j=0}^{i+1}\pi_j\left(\beta_{i+1}\beta_i-2\beta_i\beta_j+\beta_j\beta_{j-1}\right)\nonumber\\
 &=-\frac{\oF(k)}{\alpha^3\pi_{i-1}}\sum_{j=0}^i\pi_j\left\{(\beta_i-\beta_j)(\beta_i-\beta_j+2i+\beta)+2\beta_j(i+1-j)\right\}\nonumber\\
 &<0.
\end{align*}
For $i\ge k$,
 \begin{align*}\Delta^2 g(i)&=-F(k-1)\left(\frac{\oF(i+2)}{\alpha\pi_{i+1}}-2\frac{\oF(i+1)}{\alpha\pi_i}+\frac{\oF(i)}{\alpha\pi_{i-1}}\right)\nonumber\\
 &=-\frac{F(k-1)}{\alpha^3\pi_{i-1}}\left(\beta_i\beta_{i+1}\oF(i+2)-2\alpha\beta_i\oF(i+1)+\alpha^2\oF(i)\right)\nonumber\\
 &=-\frac{F(k-1)}{\alpha^3\pi_{i-1}}\sum_{j=i+2}^\infty\pi_j\left(\beta_i\beta_{i+1}-2\beta_i\beta_j+\beta_j\beta_{j-1}\right)\nonumber\\
 &=-\frac{F(k-1)}{\alpha^3\pi_{i-1}}\sum_{j=i+2}^\infty\pi_j\left\{(\beta_{j-1}-\beta_{i+1})(\beta_j-\beta_i)\right.\nonumber\\
 &\ \ \ \ \ \  \left.+(\beta+j-1)(j\beta+2ij)-(\beta+i-1)\left[i\beta+2i(j-1)\right]\right\}\nonumber\\
 &<0.
\end{align*}
For $i=k-1$, we have from (i) that
 \begin{align*}\Delta^2 g(k-1)&=\Delta g(k)-\Delta g(k-1)>0.
\end{align*}

(iii) Noting that $\beta_0=0$ and 
\begin{equation}\beta_s-\beta_r=(s-r)(\beta+s+r-1)\mbox{ for }s,r\in\Z_+,\label{lma1proofiii-01}
\end{equation} we obtain
\begin{equation}2\beta_i\beta_{k-1}-\beta_{k-1}\beta_k-\beta_{i-1}\beta_i=(k-i)(-2\beta_i+(\beta+k+i-1)(\beta_i-\beta_{k-1}))<0,
\label{lma1proofiii-02}
\end{equation}
for $1\le i\le k-1$.  We can factorise $\Delta^2 g(k-1)$ in terms of $\oF(k)$ to get
\begin{align*}
\Delta^2 g(k-1)&=\frac{F(k-1)}{\pi_k}\left(\frac{\oF(k)}{\beta_k}-\frac{\oF(k+1)}{\alpha}\right)
+\frac{\oF(k)}{\pi_{k-1}}\left(\frac{F(k-1)}{\alpha}-\frac{F(k-2)}{\beta_{k-1}}\right)\nonumber\\
&=\frac{F(k-1)}{\alpha}+\frac{\oF(k)}{\alpha^3\pi_{k-2}}\sum_{i=1}^{k-1}(2\beta_i\beta_{k-1}-\beta_{k-1}\beta_k-\beta_{i-1}\beta_i)\pi_i\nonumber\\
&\ \ \ +\frac{\oF(k)}{\alpha^3\pi_{k-2}}\left(-\pi_0+\pi_k\right)\beta_{k-1}\beta_k\\
&<\frac{F(k-1)}{\alpha}+\frac{\oF(k)}{\alpha^3\pi_{k-2}}\left(-\pi_0+\pi_k\right)\beta_{k-1}\beta_k\\
&=\frac1\alpha-\frac{\pi_0\oF(k)}{\alpha\pi_k}
<\frac{1-\pi_0}\alpha.\end{align*}

To obtain the bound $\Delta^2 g(k-1)<\frac1{\beta_{k-1}}$, we further expand
 \Ref{lma1proofiii-02} to get
$$2\beta_{k-1}\beta_i-\beta_{k-1}\beta_k-\beta_{i-1}\beta_i
=(k-i)[(i-k+1)(\beta+i+k-3)(\beta+i+k-2)-2(k-1)(\beta+k-2)]<0,$$
for $i\ge k+1$. 
We now factorise $\Delta^2 g(k-1)$ in terms of $F(k-1)$ to get
 \begin{align*}
\Delta^2 g(k-1)&=\frac{F(k-1)}{\alpha^3\pi_{k-2}}\sum_{i=k+1}^\infty(2\beta_{k-1}\beta_i-\beta_{k-1}\beta_k-\beta_{i-1}\beta_i)\pi_i+\frac{F(k-1)}{\beta_{k-1}}+\frac{\oF(k)}{\beta_{k-1}}\\
&<\frac{1}{\beta_{k-1}}.
\end{align*}

The claim of $\Delta^2 g(k-1)<\frac1{\beta+1}$ follows from $\Delta^2 g(k-1)<\frac{1-\pi_0}\alpha\wedge\frac1{\beta_{k-1}}$ for $k\ge 3$ or $\alpha\ge \beta+1$, hence it suffices to show that $\Delta^2 g_2(1)<\frac1{\beta+1}$ for $\alpha< \beta+1$. If $\alpha<\beta+1$, direct verification ensures that $\alpha(\alpha-2)-2\beta(\beta+1)<0$, hence
\begin{align*}
\Delta^2 g_2(1)&=\frac{F(1)}{\pi_2}\left(\frac{\oF(2)}{\beta_2}-\frac{\oF(3)}{\alpha}\right)
+\frac{\oF(2)}{\pi_{1}}\left(\frac{F(1)}{\alpha}-\frac{F(0)}{\beta_{1}}\right)\nonumber\\
&=\alpha^{-3}\{\alpha(\alpha+2\beta)\oF(2)-(\alpha+\beta)\beta_2\oF(3)\}\nonumber\\
&=\frac{\pi_0}{\beta_1\beta_2}(\alpha+2\beta)+\frac{(\alpha(\alpha-2)-2\beta(\beta+1))\oF(3)}{\alpha^3}\\
&<\frac{\pi_0(\alpha+2\beta)}{\beta_1\beta_2}\textcolor{red}{=\frac{\pi_0}{\beta+1}+\frac{\pi_1}{2(\beta+1)}}<\frac1{\beta+1}.\end{align*}

(iv) For $1\le i\le k-3$,
 \begin{align}\Delta^3 g(i)&=-\oF(k)\left(\frac{F(i+2)}{\alpha\pi_{i+2}}-3\frac{F(i+1)}{\alpha\pi_{i+1}}+3\frac{F(i)}{\alpha\pi_i}-\frac{F(i-1)}{\alpha\pi_{i-1}}\right)\nonumber\\
 &=-\frac{\oF(k)}{\alpha^4\pi_{i-1}}\left(\beta_{i+2}\beta_{i+1}\beta_iF(i+1)-3\alpha\beta_{i+1}\beta_iF(i)+3\alpha^2\beta_iF(i-1)-\alpha^3F(i-2)\right)\nonumber\\
 &=-\frac{\oF(k)}{\alpha^4\pi_{i-1}}\sum_{j=0}^{i+1}\pi_j\left(\beta_{i+2}\beta_{i+1}\beta_i-3\beta_i\beta_j(\beta_{i+1}-\beta_{j-1})-\beta_j\beta_{j-1}\beta_{j-2}\right).\label{lma1proof1.1}
 \end{align}
 Using \Ref{lma1proofiii-01},
 we have
 \begin{align*}
 &\beta_{i+2}\beta_{i+1}\beta_i-\beta_j\beta_{j-1}\beta_{j-2}\nonumber\\
 &=(\beta_{i+2}-\beta_j)\beta_{i+1}\beta_i+\beta_j(\beta_{i+1}-\beta_{j-1})\beta_i+\beta_j\beta_{j-1}(\beta_i-\beta_{j-2})\nonumber\\
 &=(i+2-j)(\beta+i+j-1)(\beta_{i+1}\beta_i+\beta_j\beta_i+\beta_j\beta_{j-1})+2(i+2-j)(\beta_{i+1}\beta_i-\beta_j\beta_{j-1}),
 \end{align*}
 and
 $$\beta_{i+1}-\beta_{j-1}=(i+2-j)(\beta+i+j-1),%\label{lma1proof1.3}
$$
 hence we can rewrite \Ref{lma1proof1.1} as 
 \begin{align*}
 \Delta^3 g(i)&=-\frac{\oF(k)}{\alpha^4\pi_{i-1}}\sum_{j=0}^{i+1}\pi_j(i+2-j)\left\{(\beta+i+j-1)(i+1-j)[(\beta+i+j)(\beta_i-\beta_j)+2\beta_j]\right.\nonumber\\
 &\ \ \ \ \ \ \left.+2(\beta_{i+1}\beta_i-\beta_j\beta_{j-1})\right\}\nonumber\\
 &<0.
\end{align*}

For $i\ge k$, we can argue as for $1\le i\le k-3$ to get
\begin{align}\Delta^3 g(i)&=\frac{F(k-1)}{\alpha^4\pi_{i-1}}\left\{\alpha^3\oF(i)-3\alpha^2\beta_i\oF(i+1)+3\alpha\beta_i\beta_{i+1}\oF(i+2)-\beta_{i+2}\beta_{i+1}\beta_i\oF(i+3)\right\}\nonumber\\
 &=\frac{F(k-1)}{\alpha^4\pi_{i-1}}\sum_{j=i+3}^\infty\pi_jd_{ij},\label{lma1proof1.4}
 \end{align}
where 
$$d_{ij}:=\beta_j\beta_{j-1}\beta_{j-2}-3\beta_i\beta_j(\beta_{j-1}-\beta_{i+1})-\beta_{i+2}\beta_{i+1}\beta_i.$$
We apply \Ref{lma1proofiii-01} again to get 
 \begin{align*}
 &\beta_j\beta_{j-1}\beta_{j-2}-\beta_{i+2}\beta_{i+1}\beta_i\nonumber\\
 &=(\beta_j-\beta_{i+2})\beta_{j-1}\beta_{j-2}+\beta_{i+2}(\beta_{j-1}-\beta_{i+1})\beta_{j-2}+\beta_{i+2}\beta_{i+1}(\beta_{j-2}-\beta_i)\nonumber\\
 &=(j-i-2)(\beta+i+j-1)(\beta_{j-1}\beta_{j-2}+\beta_{i+2}\beta_{j-2}+\beta_{i+2}\beta_{i+1})+2(j-i-2)(\beta_{j-1}\beta_{j-2}-\beta_{i+2}\beta_{i+1}),
 \end{align*}
 and
 \[\beta_{j-1}-\beta_{i+1}=(j-i-2)(\beta+i+j-1),
\]
therefore,
\begin{align*}
d_{ij}&=2(j-i-2)(\beta_{j-1}\beta_{j-2}-\beta_{i+2}\beta_{i+1})\\
&\ \ \ +(j-i-2)(\beta+i+j-1)\{\beta^2(j-i)(j-i-1)\\
&\ \ \ \ \ +\beta\left[\left((j-3)^2-i^2\right)\left(2(j-3-i)+1\right)+8(j-3)(j-3-i)+11(j-3)+i+6\right]\\
&\ \ \ \ \ +\left((j-3)^2-i^2\right)^2+4(j-3)((j-3)^2-i^2)+7(j-3)^2+4(j-3)\\
&\ \ \ \ \ +\left.i(j-3-i)\left(18+6(j-3)-4i\right)+5i^2+20i\right\}\\
&>0,\end{align*}
and \Ref{lma1proof1.4} ensures $\Delta^3 g(i)>0$.

 The remaining claims are straightforward consequences of (ii), namely, $\Delta^3 g(k-1)<0$ and $\Delta^3 g(k-2)>0$. \qed
 
 \section{On the large deviation of Poisson binomial}%\label{secIntroduction}
 
 Let $X_i,\ 1\le i\le n$, be independent Bernoulli random variables with $\prob(X_i=1)=1-\prob(X_i=0)=p_i\in(0,1)$, define $\lambda_j=\sum_{i=1}^np_i^j$, and write $\lambda:=\lambda_1$, $W=\sum_{i=1}^nX_i$, $W^i=W-X_i$ and $W^{ij}=W^i-X_j$ for $j\ne i$. We assume the following two conditions:
 
 \begin{description}
 \item{(Cn1)} $\lambda^2\lambda_3-2\lambda\lambda_2^2+3\lambda_2^3-2\lambda_2^2\lambda_3+\lambda_2\lambda_3-5\lambda_2\lambda_4+2\lambda_2\lambda_5+2\lambda_3^2\le 0.$
  \item{(Cn2)} $\lambda^2\lambda_3-2\lambda\lambda_2\lambda_3-\lambda_2^3+2\lambda_2^2\lambda_3-\lambda_2\lambda_3+\lambda_2\lambda_4-2\lambda_2\lambda_5+2\lambda_3^2\ge 0.$
 \end{description}
 
 When $p_i=p$, $1\le i\le n$, (Cn1) and (Cn2) are the same and they are equivalent to $p\le 1/2$. For the number of records, (Cn2) holds but (Cn1) does not hold, \textcolor{red}{ can we find another example such that (Cn1) holds but (Cn2) does not hold? }
 
 \begin{lma}\label{lma2} Assume that (Cn1) and (Cn2) hold, and $\max_{1\le i\le n}p_i\le0.5$, we have
 \begin{equation}\frac{\prob(W\ge k)}{\pi_{[k,\infty)}}\le 9\label{lma2-1}\end{equation}
 for all $k\in \Z_+$, where $\pi_{[k,\infty)}:=\sum_{i=k}^\infty \pi_i$.
 \end{lma}

\noindent{\bf Proof}
 Let $\scrB g(i):=\alpha g(i+1)-\beta_i g(i)$, and $g_k$ be the solution to the Stein equation
 \begin{equation}
 \scrB g_k(i)=\bone_{[k,\infty)}-\pi_{[k,\infty)}.\label{steineq}
 \end{equation}
 We drop the subindex $k$ from $g_k$ if there is no confusion. Then, with $\alpha=\beta\lambda+\lambda^2-\lambda_2$,
 $\beta=\lambda^2/\lambda_2-1-2\lambda+2\lambda_3/\lambda_2$, from \cite[Page~1389]{BX01} we have that
 \begin{align}
 \mean \scrB g(W)&=\sum_{i=1}^np_i^2\beta\mean \Delta g(W^i+1)+\sum_{j\ne i}p_i^2p_j^2\mean\Delta^2g(W^{ij}+1)\nonumber\\
 &\ \ \ -\sum_{j\ne i}p_ip_j(1-p_i-p_j)\mean\Delta g(W^{ij}+1). \label{prooflemma1-1}
 \end{align}
 Define $e_1=\beta\sum_{i=1}^np_i^2$, then it is a routine exercise to verify that $e_1=\sum_{j\ne i} p_ip_j(1-p_i-p_j)$. We can construct random indices $(I^+,I^-,J^-)$ such that 1) $\prob(I^+=i)=p_i^2/\lambda_2$, $\prob((I^-,J^-)=(i,j))=p_ip_j(1-p_i-p_j)/e_1$, $1\le i\ne j\le n$; 2) $(I^+,I^-,J^-)$ is independent of $\{X_i\}$. By the randomisation of the indices, we can rewrite \Ref{prooflemma1-1} as
 \begin{align}
 \mean \scrB g(W)&=e_1\mean \Delta g(W^{I^+}+1)+\sum_{j\ne i}p_i^2p_j^2\mean\Delta^2g(W^{ij}+1)\nonumber\\
 &\ \ \ -e_1\mean\Delta g(W^{I^-J^-}+1). \label{prooflemma1-2}
 \end{align} 
 Now, (Cn1) is equivalent to $\prob(X_{I^-}+X_{J^-}=0)\le \prob(X_{I^+}=0)$, hence $X_{I^-}+X_{J^-}\stg X_{I^+}$, where $\stg$ stands for `stochastically larger than or equal to'. On the other hand, (Cn2) is equivalent to $\prob(X_{I^-}+X_{J^-}=2)\le \prob(X_{I^+}=1)$. Hence (Cn1) and (Cn2) allow us to construct $(I^+,I^-,J^-)$ together such that  $\{X_{I^-}+X_{J^-}=0\}\subset\{X_{I^+}=0\}$ and $\{X_{I^-}+X_{J^-}=2\}\subset \{X_{I^+}=1\}$. We can rewrite \Ref{prooflemma1-2} as
 \begin{align}
 \mean \scrB g(W)&=e_1\mean \Delta^2 g(W^{I^-J^-}+1)\left(\bone_{[X_{I^-}+X_{J^-}=1,X_{I^+}=0]}+\bone_{[X_{I^-}+X_{J^-}=2]}\right)
\nonumber \\
 & \ \ \ +\sum_{j\ne i}p_i^2p_j^2\mean\Delta^2g(W^{ij}+1)\nonumber\\
 &\le e_1\Delta^2 g_k(k-1)\textcolor{red}{\E}\left[\bone_{[W^{I^-J^-}=k-2]}\left(\bone_{[X_{I^-}+X_{J^-}=1,X_{I^+}=0]}+\bone_{[X_{I^-}+X_{J^-}=2]}\right)\right]
 \nonumber\\ & \ \ \ +\Delta^2 g_k(k-1)\sum_{j\ne i}\prob(W^{ij}=k-2)p_i^2p_j^2\nonumber\\
 &\le e_1\Delta^2 g_k(k-1)\textcolor{red}{\E}\left[\bone_{[W^{I^-J^-}=k-2]}\left(\bone_{[X_{I^-}+X_{J^-}=1]}+\bone_{[X_{I^-}+X_{J^-}=2]}\right)\right]
 \nonumber\\ & \ \ \ +\Delta^2 g_k(k-1)\sum_{j\ne i}\prob(W^{ij}=k-2)p_i^2p_j^2\nonumber\\
 &=\Delta^2 g_k(k-1)\sum_{j\ne i}\prob(W^{ij}=k-2)p_ip_j[(p_i+p_j)(1-p_i)(1-p_j)].
 \label{prooflemma1-3}
 \end{align}
However,
$$\sum_{i\ne j}p_ip_j\prob(W^{ij}=k-2)=\mean \left\{W(W-1)\bone_{[W=k]}\right\}=k(k-1)\prob(W=k),$$
\Ref{prooflemma1-3} implies
 \begin{align}
 \mean \scrB g(W)
 &\le \max_{i_1\ne i_2}[(p_{i_1}+p_{i_2})(1-p_{i_1})(1-p_{i_2})]\Delta^2 g_k(k-1)\sum_{j\ne i}\prob(W^{ij}=k-2)p_ip_j\nonumber\\
 &\le \frac8{27}\Delta^2 g_k(k-1)k(k-1)\prob(W=k)
 \label{prooflemma1-4}\\
 &\le \frac8{27}\Delta^2 g_k(k-1)k(k-1)\prob(W\ge k)
 \label{prooflemma1-5}
 \end{align}
 For $k\ge 3$, we combine \Ref{prooflemma1-5} and Lemma~\ref{lma1} with $\beta_{k-1}$ to get
 $$\prob(W\ge k)-\pi_{[k,\infty)}\le \frac8{27}k(k-1)\prob(W\ge k)/\beta_{k-1}\le \frac89\prob(W\ge k),$$
 which implies \Ref{lma2-1} for $k\ge 3$. For $k=2$, we combine \Ref{prooflemma1-5} and Lemma~\ref{lma1} with $1/(\beta+1)$ to get
  $$\prob(W\ge 2)-\pi_{[2,\infty)}\le \frac{16}{27}\prob(W\ge 2)/(\beta+1)\le \frac{16}{27}\prob(W\ge 2),$$
  which yields
  $\prob(W\ge 2)/\pi_{[2,\infty)}<27/11.$ For $k=1$, since $1-x>e^{-2x}$ for $x<0.796$, we have $\prod_{i=1}^n(1-p_i)>e^{-2\lambda}$ provided $\max_{1\le i\le n}p_i<0.796$. On the other hand, $\alpha=\beta\lambda+\lambda^2-\lambda_2>\beta\lambda$, we have $\alpha/\beta>\lambda$ and
  \begin{align*}
  \frac{\prob(W\ge 1)}{\pi_{[1,\infty)}}&=\frac{1-\prod_{i=1}^n(1-p_i)}{1-\pi_0}\\
  &=\left(1-\prod_{i=1}^n(1-p_i)\right)\left(1+\left(\sum_{i=1}^\infty\prod_{1\le j\le i}\frac \alpha{\beta_j}\right)^{-1}\right)\\
  &\le (1-e^{-2\lambda})(1+(\alpha/\beta)^{-1})\le (1-e^{-2\lambda})(1+\lambda^{-1})\\
  &<3.
  \end{align*}
  \qed
 
 {\bf Problem} We need to get a better estimate of \Ref{prooflemma1-4} to \Ref{prooflemma1-5}.
 
 %%%%%%%%%%%%%%%%%%%%%%%%%%%%%
%%%%%%% References   %%%%%%%%
%%%%%%%%%%%%%%%%%%%%%%%%%%%%%

\def\ac{{Academic Press}~}
\def\aap{{Adv. Appl. Prob.}~}
\def\ap{{Ann. Probab.}~}
\def\anap{{Ann. Appl. Probab.}~}
\def\eljp{{\it Electron.\ J.~Probab.\/}~} 
\def\jap{{J. Appl. Probab.}~}
\def\jws{{John Wiley~$\&$ Sons}~}
\def\ny{{New York}~}
\def\ptrf{{Probab. Theory Related Fields}~}
\def\sp{{Springer}~}
\def\spa{{Stochastic Process. Appl.}~}
\def\sv{{Springer-Verlag}~}
\def\tpa{{Theory Probab. Appl.}~}
\def\zw{{Z. Wahrsch. Verw. Gebiete}~}

 \begin{thebibliography}{9}

 \bibitem[Brown \& Xia~(2001)]{BX01}Brown, T. C. \& Xia, A.~(2001). Stein's method and birth-death processes. \emph{\ap}1373--1403.
 
 \end{thebibliography}
\end{document}










